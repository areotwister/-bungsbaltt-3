\keywords{Memory Map}
\history{e-sys-ws24}
\begin{exercise}
	Sie werden unter Verwendung von LOAD- und STORE-Befehlen immer wieder
	abwechselnd mit einem Programm auf Adressen im Adressraum des EPROM (00), der 
	UART (01) und des SRAM (1*)	zugreifen müssen.
	
	Daher muss das Datensegmentregister \verb_DS_ immer wieder neu belegt werden 
	(für die Ergänzung der 22-Bit Adressen mit einem Präfix). 
	
	Beispiel UART: Offensichtlich ist es nicht möglich, einfach $w = 010...0$ (32 Stellen)
	per \texttt{LOADI DS i} in \texttt{DS} zu laden, da \texttt{i} nur 22 Stellen erlaubt.

	Eine Lösung wäre beispielsweise, die obersten 22 Bit von 

	{\color{red}\verb_0100000000000000000000_}\verb_0000000000_

	zu laden, per:

	\verb_LOADI DS _{\color{red}\verb_0100000000000000000000_}

	und dann um $10$ Stellen nach links zu shiften (per Multiplikation mit $2^{10}$).

	Wir nehmen zudem an, dass nicht nur die obersten Bits interessant sind,
	sondern die gesamte 32-Bit Konstante.
	D.h. die unteren 10 Bit müssen noch nachgeladen werden (z.B. per \texttt{ORI}).

	\emph{Bonus:} Besondere Vorsicht ist geboten, wenn Befehle geladen werden,
	die an der vordersten Bit-Stelle eine 1 haben. Beim Shift per Multiplikation wird 
	Zweier-Komplement angenommen und die vorderste Bit-Stelle wird u.U. immer 0 bleiben,
	da zunächst von der Multiplikation mit einer positiven Zahl ausgegangen wird.
	Können Sie sich eine Lösung vorstellen?

	\textbf{Arbeitsanweisung:}

	Schreiben Sie ein RETI-Programm \texttt{constants.reti}, das das 32-Bit-Wort 
	
	$-1431655766 = [10101010101010101010101010101010]_2$

	lädt und im \texttt{ACC} speichert.

	\textbf{Bitte testen Sie Ihr Programm mit dem RETI-Interpreter!} Nutzen Sie hierzu folgenden Befehl:
	\begin{verbatim}
		$ reti_interpreter -d constants.reti
	\end{verbatim}
	\nicebox{\textwidth}{
	\textbf{Tipp:} Wenn Sie Schritt für Schritt durch das Programm `steppen' (per (n)ext instruction), wird Ihnen auffallen,
	dass zunächst mal ein Initialisierungsprogramm auf dem EPROM abläuft (technisch notwendig, automatisch generiert).
	Das braucht Sie für das Ausführen Ihres Programmes und die Überprüfung der Registerinhalte zunächst nicht zu interessieren.
	Entweder sie setzen einen entsprechenden Breakpoint oder lassen das EPROM-Programm einfach ablaufen.
	}

\end{exercise}

\begin{solution}
Bezüglich Bonus: Man kann zunächst mal \texttt{LOADI <Zielreg> -1} machen, dann stehen nur 1en drin.
Dann vielleicht per XOR a la \texttt{OPLUSI <Zielreg> 111..11} 21 Einsen zur 0 machen.
Dann die \textbf{--- nach dem ersten Bit ---} 21 vordersten Bit des Wortes per \texttt{ORI} in \texttt{<Zielreg>} schreiben.
10-mal shiften usw.

Vollständig korrekte Lösung {\color{blue}(???)}, Annahme: Konstante $-1431655766 = 10101010101010101010101010101010$.
{\color{blue}RETI-Interpreter zeigt Registerinhalte nicht als Zweierkomplement-Integer an. (Gibt es eine Option?)
Kann sein, dass ich mich in den Stellen vertan habe.}
\begin{verbatim}
LOADI ACC -1
OPLUSI ACC 2097151
ORI ACC 699050
MULTI ACC 1024
ORI ACC 682
JUMP 0
\end{verbatim}

\end{solution}

\begin{sketch}


\end{sketch}

