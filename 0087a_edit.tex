\documentclass{article}
\usepackage{xcolor}
\usepackage[parfill]{parskip}

\begin{document}
	Sie werden unter Verwendung von LOAD- und STORE-Befehlen immer wieder
	abwechselnd mit einem Programm auf Adressen im Adressraum des EPROM (00), der 
	UART (01) und des SRAM (1*)	zugreifen müssen.
	
	Daher muss das Datensegmentregister \verb_DS_ immer wieder neu belegt werden 
	(für die Ergänzung der 22-Bit Adressen mit einem Präfix). 
	
	Beispiel UART: Offensichtlich ist es nicht möglich, einfach $w = 010...0$ (32 Stellen)
	per \texttt{LOADI DS i} in \texttt{DS} zu laden, da \texttt{i} nur 22 Stellen erlaubt.

	Eine Lösung wäre beispielsweise, die obersten 22 Bit von 

	{\color{red}\verb_0100000000000000000000_}\verb_0000000000_

	zu laden, per:

  \verb_LOADI DS _{\color{red}\verb_1048576_}\verb_ # (_{\color{red}\verb_0100000000000000000000_}\verb_)_

	und dann um $10$ Stellen nach links zu shiften (per Multiplikation mit $2^{10}$):

  \verb_MULTI DS 1024_

	Wir nehmen zudem an, dass nicht nur die obersten Bits interessant sind,
	sondern die gesamte 32-Bit Konstante.
	D.h. die unteren 10 Bit müssen noch nachgeladen werden (z.B. per \texttt{ORI}).

	\textbf{Arbeitsanweisung:}

	Schreiben Sie ein RETI-Programm \texttt{constants.reti}, welches das 32-Bit-Wort 
	
	$1431655765 = [01010101010101010101010101010101]_2$

	im \texttt{ACC} speichert.

  \emph{Bonus:} Besondere Vorsicht ist geboten, wenn \textcolor{green}{Konstanten (z.B. ein Befehl)} geladen werden,
	die an der vordersten Bit-Stelle eine 1 haben. Beim Shift per Multiplikation mit einer positiven Zahl wird 
	Zweier-Komplement angenommen und die vorderste Bit-Stelle wird immer 0 bleiben,
  da \textcolor{green}{bei der Multiplikation mit einer positiven Zahl nicht plötzlich eine negative Zahl (das vorderste Bit auf 1 gesetzt) als Ergebnis rauskommen kann, sofern nicht gezielt undefiniertes Verhalten bei einem Overflow ausgenutzt wird}.

	Schreiben Sie als Bonus auf ihr RETI-Programm in \texttt{constants.reti} folgend und am besten durch einen Kommentar \verb|# bonus| abgegrenzt ein Programm, welches das 32-Bit-Wort 
	
	$-1431655766 = [10101010101010101010101010101010]_2$

	im \texttt{ACC} speichert.

  {\LARGE\bfseries\textcolor{red}{Bitte}} testen Sie Ihr Programm mit dem \textbf{RETI-Interpreter}! Nutzen Sie hierzu den folgenden Befehl:
	\begin{verbatim}
		$ reti_interpreter -d -b constants.reti
	\end{verbatim}

  {\color{green}
Mithilfe der Kommandozeilenoption \verb|-d| ist der RETI-Interpreter in der Lage das Programm zu \textbf{debuggen}, d.h. er zeigt die Speicher- und Registerinhalte nach Ausführung eines jeden Befels an, zwischen denen der Benutzer sich mittels \verb|n| und dann \verb|Enter| bewegen kann. Wird \verb|INT 3| in das RETI-Programm geschrieben stellt dies einen Breakpoint dar, zu dem mittels \verb|c| und dann \verb|Enter| gesprungen werden kann. In der \textbf{neusten Version} der RETI-Interpreters stehen alle Aktionen, die Sie in diesem Debug-Modus ausführen können in der untersten Zeile des Text-User-Interfaces (TUI).

\textit{Tipp:} Mit der Kommandozeilenoption \verb|-b| werden alle Registerinhalte, Memoryinhalte und Immediates im Binärsystem angezeigt, damit lässt sich beim debuggen leichter das ganze Shiften nachvollziehen.
}

	% \nicebox{\textwidth}{
	% \textbf{Tipp:} Wenn Sie Schritt für Schritt durch das Programm `steppen' (per (n)ext instruction), wird Ihnen auffallen,
	% dass zunächst mal ein Initialisierungsprogramm auf dem EPROM abläuft (technisch notwendig, automatisch generiert).
	% Das braucht Sie für das Ausführen Ihres Programmes und die Überprüfung der Registerinhalte zunächst nicht zu interessieren.
	% Entweder sie setzen einen entsprechenden Breakpoint oder lassen das EPROM-Programm einfach ablaufen.
	% }


% Bezüglich Bonus: Man kann zunächst mal \texttt{LOADI <Zielreg> -1} machen, dann stehen nur 1en drin.
% Dann vielleicht per XOR a la \texttt{OPLUSI <Zielreg> 111..11} 21 Einsen zur 0 machen.
% Dann die \textbf{--- nach dem ersten Bit ---} 21 vordersten Bit des Wortes per \texttt{ORI} in \texttt{<Zielreg>} schreiben.
% 10-mal shiften usw.
%
% Vollständig korrekte Lösung {\color{blue}(???)}, Annahme: Konstante $-1431655766 = 10101010101010101010101010101010$.
% {\color{blue}RETI-Interpreter zeigt Registerinhalte nicht als Zweierkomplement-Integer an. (Gibt es eine Option?)
% Kann sein, dass ich mich in den Stellen vertan habe.}
% \begin{verbatim}
% LOADI ACC -1
% OPLUSI ACC 2097151
% ORI ACC 699050
% MULTI ACC 1024
% ORI ACC 682
% JUMP 0
% \end{verbatim}
\begin{verbatim}
# output: 1431647027 -1431647028
INT 3 # just a breakpoint, use c in debug mode -d to directly jump here
# Konstante, bei der bit 31 auf 0 gesetzt ist
# Als Beispiel hier: 0b01010101_01010101_00110011_00110011
LOADI ACC 349525 # 0b01010101_01010101
MULTI ACC 65536
ORI ACC 13107 # 0b110011_00110011
INT 0
INT 3 # just a breakpoint, use c in debug mode -d to directly jump here
# Konstante, bei der bit 31 auf 1 gesetzt ist
# Als Beispiel hier: 0b10101010_10101010_11001100_11001100
# Das geht nicht so einfach, weil man durch Multiplizieren zweier positiver
# Zahlen nicht plötzlich etwas Negatives (das MSB auf 1) rausbekommen kann,
# ohne sich auf undefined behaviour verlassen zu müssen
LOADI ACC -21846 # 0b101010_10101010 - 2**15 oder 0b11111_10101010_10101010 - 2**21
MULTI ACC 65536
ORI ACC 52428 # 0b11001100_11001100
INT 0
JUMP 0
\end{verbatim}


\end{document}
